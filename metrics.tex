\documentclass{article}
\usepackage{amsmath}
\begin{document}
\section{Definition of metrics}
\subsection{Distance metric}
The essential features of the distance metric for swarms, presented in (Eliot et al. 2018), can be summarised as follows:

$$
\psi_d(S) = \mu_d(S) \pm \sigma_d(S)
$$

where $\mu_d(S)$ is the mean distance over all agents $b \in S$, between $b$ and its cohesion neighbours, given by:

$$
\mu_d(S) = \frac{\sum_{b \in S} \sum_{b' \in n_c(b)}\, \lVert\vec{b b'}\rVert}{\sum_{b \in S}\,\big\lvert n_c(b)\big\rvert}
$$

and $\sigma_d(S)$ is the standard deviation from the mean:

$$
\sigma_d(S) = \sqrt{\frac{\sum_{b \in S} \sum_{b' \in n_c(b)}\, \left(\lVert\vec{b b'}\rVert - \mu_d(S)\right)^2}{\sum_{b \in S}\,\big\lvert n_c(b)\big\rvert}}
$$
\subsection{Cohesion/repulsion metric}
The essential features of the cohesion/repulsion metric for swarms, adapted from (Eliot et al. 2018), can be summarised as follows:

$$
\psi_p(S) = \mu_p(S) \pm \sigma_p(S)
$$

where $\mu_p(S)$ is the mean of the adjusted magnitude values of the weighted cohesion/repulsion vectors of all agents, induced by their cohesion/repulsion neighbours. For each cohesion/repulsion vector, a positive value is derived from the magnitude of the vector if the cohesion component of the
vector dominates, but a negative value is derived if the repulsion component dominates.

We define some helper functions, $v_{cr}$ and $P$, to 
aid the specifications of $\mu_p$ and $\sigma_p$:

$$
v_{cr}(b) = k_c v_c(b) + k_r v_r(b)
$$

$$
P(b) = \left\{ \begin{array}{ll}
                \lVert v_{cr}(b) \rVert & \quad k_c v_c(b) > k_r v_r(b) \\
                -\lVert v_{cr}(b) \rVert & \quad \mathrm{otherwise}
              \end{array}
       \right.
$$

$v_{cr}(b)$ gives the weighted cohesion/repulsion vector for $b$ and $P(b)$ gives the value derived from the magnitude of this vector. Now we can define the mean and standard deviation.

$$
\mu_p(S) = \frac{\sum_{b \in S} P(b)}{D}
$$

and $\sigma_p(S)$ is the standard deviation from the mean:

$$
\sigma_p(S) = \sqrt{\frac{\sum_{b \in S}\, (P(b) - \mu_p(S))^2}{D}}
$$

We still need to consider the definition of the denominator, $D$, here. (Eliot et al. 2018) defines $D$ like this:

$$
D = \sum_{b \in S}\, \big\lvert n_c(b) \big\rvert + \big\lvert n_r(b) \big\rvert
$$

This seems to me to be over-counting agents. Remember that each agent $b \in S$ has at most one cohesion/repulsion vector as defined above. This has been scaled already by the reciprocal of the number of its cohesion and repulsion neighbours (see the definitions of $v_c(b)$ and $v_r(b)$ earlier). In calculating $\mu_p(S)$ and $\sigma_p(S)$, we should be dividing by at most $\lvert S \rvert$ but the value of $D$, as defined above, may be as big as $2(\lvert S\rvert^2 - \lvert S\rvert)$, clearly too big!. It might be argued that even $\lvert S \rvert$ may be too big, since $S$ may include agents that are isolated and not participating in the cohesion/repulsion structure of the swarm and, therefore, should not be counted. In this case, we could define $D$ as

$$
D = \bigg\lvert \bigcup_{b \in S} (n_c(b) \cup n_r(b)) \bigg\rvert
$$

I think it's reasonable to consider that the cohesion/repulsion structure is a property of the whole swarm $S$, whether or not it contains isolated agents, and, in the following, I take $D$ to be

$$
D = \lvert S \rvert
$$

\section{Implementation of metrics}

\subsection{Distance metric}
\begin{verbatim}
@jit(nopython=True, fastmath=True)
def mu_sigma_d(mag, ecb):
    n_agents = mag.shape[0]
    msum = 0; msum_sq = 0; nsum = 0
    for i in prange(n_agents):
        for j in range(i):
            if mag[j, i] <= ecb[j, i]:
                msum += mag[j, i]
                msum_sq += mag[j, i] **2
                nsum += 1
            if mag[i, j] <= ecb[i, j]:
                msum += mag[i, j]
                msum_sq += mag[i, j] **2
                nsum += 1
    mu_d = msum / nsum
    mu_d_sq = msum_sq / nsum
    var_d = mu_d_sq - mu_d ** 2
    sigma_d = np.sqrt(var_d)
    return mu_d, sigma_d
\end{verbatim}

\subsection{Cohesion/repuslion metric}
\begin{verbatim}
def mu_sigma_p(b):
    vcr_x = b[COH_X] + b[REP_X]                                 # the weighted cohesion/repulsion vector of every agent
    vcr_y = b[COH_Y] + b[REP_Y]
    vcr_mag = np.hypot(vcr_x, vcr_y)                            # the magnitude of the weighted cohesion/repulsion vector of every agent
    vc_mag = np.hypot(b[COH_X], b[COH_Y])                       # the magnitude of the cohesion component of the cohesion/repulsion vector
    vr_mag = np.hypot(b[REP_X], b[REP_Y])                       # the magnitude of the repulsion component of the cohesion/repulsion vector
    P = np.where(vc_mag > vr_mag, vcr_mag, -vcr_mag)            # the implementation of P as defined
    n_agents = b.shape[1]                                       # the total number of agents in the swarm
    mu_p = np.sum(P) / n_agents                                 # the mean
    sigma_p = np.sqrt(np.sum((P - mu_p) ** 2) / n_agents)       # the standard deviation
    return mu_p, sigma_p
\end{verbatim}
\end{document}
